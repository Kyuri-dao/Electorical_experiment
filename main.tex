\documentclass{article}
\usepackage{graphicx} % Required for inserting images

\title{重ね合わせの理・テブナンの定理の諸式}
\author{Kyuri-dao}
\date{June 2025}

\begin{document}

\maketitle

\section{はじめに}
これはWordで数式を張り付けるのがめんどくせぇ人に向けたテキストです.\\
LaTeXで書かれているので,コードそのものが欲しい人は僕のリポジトリまで来てください.

\section{重ね合わせの理}

\begin{center}
    電流$I_1'$,$I_2'$,$I_3'$の式.
\end{center}



\begin{equation}
    I_1' = \frac{(R_1 + R_2)E_1}{R_1R_2+R_2R_3+R_3R_1}
\end{equation}

\begin{equation}
    I_2' = \frac{R_3E_1}{R_1R_2+R_2R_3+R_3R_1}
\end{equation}

\begin{equation}
    I_3' = \frac{R_2E_1}{R_1R_2+R_2R_3+R_3R_1}
\end{equation}

\begin{center}
    電流$I_1''$,$I_2''$,$I_3''$の式.    
\end{center}


\begin{equation}
    I_2'' = \frac{(R_1+R_3)E_2}{R_1R_2+R_2R_3+R_3R_1}    
\end{equation}

\begin{equation}
    I_3'' = \frac{R_3E_2}{R_1R_2+R_2R_3+R_3R_1}
\end{equation}

\begin{equation}
    I_1'' = \frac{R_1E_2}{R_1R_2+R_2R_3+R_3R_1}
\end{equation}

\begin{center}
    代数和の式
\end{center}

\begin{equation}
    I_1 = I_1' + (-I_1'')  \quad I_2 = (-I_2') + I_2'' \quad I_3 = I_3' + I_3''
\end{equation}

\section{テブナンの定理}

\begin{center}
    \subsection{$I_1$を求める}
\end{center}

\begin{equation}
    V_0 = \frac{R_3}{R_2+R_3}E_2
\end{equation}

\begin{equation}
    R_0 = \frac{R_2R_3}{R_2+R_3} 
\end{equation}

\begin{equation}
    I_1 = \frac{E_1-V_0}{R_1+R_0}
\end{equation}

\begin{center}
    \subsection{$I_2$を求める}
\end{center}

\begin{equation}
    V_0 = \frac{R_3}{R_1+R_3}E_1
\end{equation}

\begin{equation}
    R_0 = \frac{R_1R_3}{R_1+R_3}
\end{equation}

\begin{equation}
    I_2 = \frac{E_2-V_0}{R_2+R_0}
\end{equation}

\begin{center}
    \subsection{$I_3$を求める}
\end{center}

\begin{equation}
    V_0 = E_2 - R_2I_3' = \frac{R_1E_2+R_2E_1}{R_1+R_2}
\end{equation}

\begin{equation}
    R_0 = \frac{R_1R_2}{R_1+R_2}
\end{equation}

\begin{equation}
    I_3 = \frac{R_0}{R_3+R_0}
\end{equation}


\end{document}
